\documentclass{article}
\usepackage[left=3cm,right=3cm,top=2cm,bottom=2cm]{geometry} % page
                                                             % settings
\usepackage{amsmath} % provides many mathematical environments & tools
\usepackage[spanish]{babel}
\usepackage[doument]{ragged2e}

% Images
\usepackage{graphicx}
\usepackage{float}

% Code
\usepackage{listings}
\usepackage{xcolor}
\definecolor{gray}{rgb}{0.5,0.5,0.5}
\newcommand{\n}[1]{{\color{gray}#1}}
\lstset{numbers=left,numberstyle=\small\color{gray}}

\selectlanguage{spanish}
\usepackage[utf8]{inputenc}
\setlength{\parindent}{0mm}

\begin{document}

\title{Recorrido en postorder iterativo de un árbol binario}
\author{David Cabezas Berrido}
\date{\today}
\maketitle

\begin{justify}
  
\end{justify}

\begin{verbatim}
template <typename T>
typename bintree <T>::postorder_iterator&
bintree<T>::postorder_iterator::operator++(){

  if(elnodo.null())  // El nodo es nulo
    return *this;

  if(elnodo.parent().null())
    elnodo = typename bintree<T>::node();  // Estamos en la raíz, fin del recorrido

  else if(elnodo.parent().right().null() ||  // Somos el hijo a la derecha del padre
	  elnodo.parent().right() == elnodo) // o somos el hijo a la izquierda y no hay hijo a la derecha
    elnodo = elnodo.parent();  // Pasamos al padre

  else{                                // Somos el hijo a la izquierda y hay hijo a la derecha

    elnodo = elnodo.parent().right();  // Pasamos al hijo a la derecha, y bajamos

    while(!elnodo.left().null() || !elnodo.right().null()){  // Hasta encontrar una hoja

      if(!elnodo.left().null())  // Siempre bajamos por el hijo de la izquierda (si lo hay)
        elnodo = elnodo.left();

      else
        elnodo = elnodo.right();  // En caso negativo bajamos por el de la derecha

    }
  }

  return *this;
}  
\end{verbatim}

\end{document}