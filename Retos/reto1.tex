\documentclass{article}
\usepackage[left=3cm,right=3cm,top=0cm,bottom=2cm]{geometry} % page settings
\usepackage{amsmath} % provides many mathematical environments & tools
\usepackage[spanish]{babel}
\usepackage[doument]{ragged2e}
\usepackage{multirow}

\selectlanguage{spanish}
\usepackage[utf8]{inputenc}
\setlength{\parindent}{0mm}

\begin{document}

\title{Estructura de Datos: Reto 1}
\author{Patricia Córdoba Hidalgo y David Cabezas Berrido}
\date{\today}
\maketitle



\section*{Ejercicio 1.}

Primer código:
\begin{justify}
    La primera línea es $O(1)$. El bucle \textit{do while} itera n
    veces; las líneas 3 y 8 son $O(1)$, al igual que las líneas 4, 5 y
    6, analizadas individualmente, sin embargo, al estar dentro del
    bucle \textit{while}, aplicamos la regla de la multiplicación y
    quedan como $O(\log_{2}{n})$, ya que j va tomando valores de
    pontencias de 2. Luego este es el resultado de aplicar la regla de
    la suma a todo el bucle \textit{while} y las líneas 3 y
    8. Finalmente, aplicamos la regla de la multiplicación con el
    bucle \textit{do while} y el código completo es $O(n\log_{2}{n})$.
\end{justify}

Segundo código:
\begin{justify}
  La primera línea es $O(1)$. El bucle \textit{do while} itera n
  veces; las líneas 3 y 8 son $O(1)$, al igual que las líneas 4, 5 y
  6, analizadas individualmente, sin embargo, al estar dentro del
  bucle \textit{while}, el cual es no homogéneo, resolvemos la suma de
  la serie para calcular la eficiencia.
\end{justify}

\[
  \sum\limits_{i=2}^{n}\log_2i=\log_2\prod\limits_{i=2}^ni=\log_2n!
  \]

\begin{justify}
  Entonces, el código completo es $O(\log_2n!).$
\end{justify}

\section*{Ejercicio 2.}

\begin{justify}
  Básicamente, el ejercicio consiste en despejar $n$ de la ecuación
  $f(n)=t$ donde $t$ en cada caso es el número de microsegundos que
  hay en un segundo, hora, semana, año o 1000 años.
\end{justify}

\begin{table}[!htbp]
\centering
\label{my-label}
\begin{tabular}{|c|l|l|l|l|l|}
\hline
\multirow{2}{*}{\textit{\textbf{f(n)}}} & \multicolumn{5}{c|}{\textit{\textbf{t}}}   \\ \cline{2-6} 
                                        & 1 sg. & 1h. & 1 semana & 1 año & 1000 años \\ \hline
$\log_2n$                               & $2^{10^6}$ & $2^{3.6*10^9}$ & $2^{6.048*10^{11}}$ & $2^{3.1536*10^{13}}$ & $2^{3.1536*10^{16}}$ \\ \hline
$n$                                     & $10^6$ & $3.6*10^9$ & $6.048*10^{11}$ & $3.1536*10^{13}$ & $3.1536*10^{16}$ \\ \hline
$n\log_2n$                              & $62746$ & $1.33378*10^8$ & $1.77631*10^{10}$ & $7.97634*10^{11}$ & $6.41137*10^{14}$ \\ \hline
$n^3$                                   & $100$ & $1532$ & $8456$ & $31593$ & $315938$ \\ \hline
$2^n$                                   & $19$ & $31$ & $39$ & $44$ & $54$ \\ \hline
$n!$                                    & $9$ & $12$ & $14$ & $16$ & $18$ \\ \hline
\end{tabular}
\end{table}  

\end{document}
